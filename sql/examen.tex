\documentclass[11pt,oneside,a4paper]{article} %draft
\usepackage[latin1]{inputenc}
%\usepackage{a4wide}
%\usepackage{beton}
%\usepackage{amsmath}
\usepackage{hyperref}
\usepackage{graphicx}
\usepackage{color}
\usepackage{amssymb}
\usepackage{hyperref}
\usepackage{cancel}
\usepackage{verbatim}
\usepackage{latexsym}
\usepackage[spanish]{babel}
\usepackage{upgreek}
\usepackage{pdfpages}
\usepackage[a4paper,margin=2cm]{geometry}
%\pagestyle{empty}

\begin{document}
\parindent 0mm
\parskip 4mm

\topmargin 10mm
\textheight 228mm

\AddToShipoutPicture{\includegraphics[width=\paperwidth,height=\paperheight,
keepaspectratio]{fondo}}

\newcommand{\hs}{\hspace{1mm}}
\newcommand{\G}{ \textrm{\c{G}}}
\newcommand{\ljoin}{ \ltimes }
\newcommand{\rjoin}{ \rtimes }
\newcommand{\fjoin}{ ]\mkern-9mu\times\mkern-9mu [ }
\newcommand{\tbl}[1]{$#1$}
\newcommand{\cmp}[1]{$#1$}

%\tableofcontents
%\newpage

%-----------------------------------------------------------
\begin{comment}
%\section{Administraci�n de bases de datos}
\section{Consultas: implementar �lgebra relacional en SQL}

Nombre : \hrulefill\hrulefill \hs\hs DNI: \hrulefill \\

Implementa en SQL el �lgebra relacional de estas consultas e intenta verbalizar qu� pretende obtener cada una.
\begin{enumerate} \item 

DGPRODUCTIONS GARDEN	4000,00	6165,00
Gardening Associates	10926,00	10926,00
Gerudo Valley	81849,00	81849,00
Tendo Garden	23794,00	23794,00
Beragua [...] 

\vspace{5cm}

\item 

DGPRODUCTIONS GARDEN

\vspace{5cm}

\item 

DGPRODUCTIONS GARDEN
Gardening Associates
Gerudo Valley
Tendo Garden
Camunas Jardines S.L.
Flores Marivi
Jardineria Sara
El Jardin Viviente S.L

\vspace{5cm}

\item 

986,00

\vspace{5cm}

\item 

73510,00

\vspace{5cm}

\item 

Plantas vistosas para la decoraci�n del jard�n

\vspace{5cm}

\item 

Plantas vistosas para la decoraci�n del jard�n

\vspace{5cm}

\item 

LON-UK
PAR-FR
TOK-JP

\vspace{5cm}

\item 

FR-17

\vspace{5cm}

\item 

FR-11

\vspace{5cm}

\item 

FR-11

\vspace{5cm}

\item 

NULL

\vspace{5cm}

\item $c1\gets\prod_{CodigoPedido}(
\sigma_{Estado=Rechazado}(pedidos)
)$

$c3\gets c1\bigotimes_{CodigoPedido}detallepedido
\bigotimes_{CodigoProducto}productos$

$c4\gets_{Gama}\G_{Gama,s\gets suma(Cantidad)}(c3)$

$m\gets\G_{max(s)}(c4)$

$\prod_{Gama}(
\sigma_{s=m}(c4)
)$

Frutales

\vspace{5cm}

\item 

BCN-ES	1	9
BOS-USA	0	2
MAD-ES	3	3
PAR-FR	2	0
SFC-USA	1	3
SYD-AU	2	5
TAL-ES	0	5

\vspace{5cm}

\item $c1\gets_{Gama}\G_{Gama,s\gets suma(Cantidad)}(
detallepedidos\bigotimes_{CodidoProducto}productos)$

$m\gets\G_{max(s)}(c1)$

$\prod_{Gama}(\sigma_{s=m}(c1))$

$c3\gets\prod_{CodigoProducto}(
c2\bigotimes_{Gama}productos
)$

$\G_{suma(Cantidad*PrecioUnidad)}(
c3\bigotimes_{CodigoPedido}detallepedido
)$

140194,00

\vspace{5cm}

\item 

Gerudo Valley

\vspace{5cm}

\item 

FR-17	Rosal bajo 1ª -En maceta-inicio brotaci�n

\vspace{5cm}

\item 

OR-247	Trachycarpus Fortunei

\vspace{5cm}

\item 

New York

\vspace{5cm}

\item $c1\gets\prod_{CodigoPedido}(
\sigma_{Estado=Rechazado}(pedidos)
)$

$c2\gets_{CodigoProducto}\G_{CodigoProducto,s\gets suma(Cantidad)}(
c1\bigotimes_{CodigoPedido}detallepedido
)$

$m\gets\G_{max(s)}(c2)$

$c3\gets\prod_{CodigoProducto}(
\sigma_{s=m}(c2)
)$

$c4\gets\prod_{Gama}(
c3\bigotimes_{CodigoProducto}productos
)$

$\prod_{DescripcionTexto}(
c4\bigotimes_{Gama}gamaproductos
)$

�rboles peque�os de producci�n frutal

\vspace{5cm}

\item 

Gerudo Valley

\vspace{5cm}

\item 

�rboles peque�os de producci�n frutal

\vspace{5cm}

\item 

Gerudo Valley

\vspace{5cm}

\item 

OR-247	Trachycarpus Fortunei

\vspace{5cm}

\item 

38	El Jardin Viviente S.L	Justin	Smith	2 8005-7161	2 8005-7162	176 Cumberland Street The rocks	(null)	Sydney	Nueva Gales del Sur	Australia	2003	31	800 [...] 

\vspace{5cm}

\item 

Boston

\vspace{5cm}

\item $c1\gets\prod_{CodigoProducto}(
\sigma_{Gama=Frutales}(productos)
)$

$c2\gets\prod_{CodigoProducto,CodigoPedido}(detalleproducto)$

$c3\gets\prod_{CodigoPedido,CodigoCliente}(pedidos)$

$c4\gets\prod_{CodigoCliente,
CodigoEmpleado\gets CodigoEmpleadoRepVentas
}(clientes)$

$c5\gets\prod_{CodigoEmpleado,CodigoOficina}(empleados)$

$c6\gets\prod_{CodigoOficina,Ciudad}(oficinas)$

$c7\gets 
c1\bigotimes_{CodigoProducto}c2
\bigotimes_{CodigoPedido}c3
\bigotimes_{CodigoCliente}c4
\bigotimes_{CodigoEmpleado}c5
\bigotimes_{CodigoOficina}c6
$

$c8\gets
_{Ciudad,CodigoOficina}\G_{Ciudad,CodigoOficina,suma(Cantidad)}(c7)$

$m\gets\G_{max(s)}(c8)$

$\prod_{Ciudad}(
\sigma_{s=m}(c8)
)$

Boston

\vspace{5cm}

 \end{enumerate}\newpage

\end{comment}
%-----------------------------------------------------------


%-----------------------------------------------------------
%\begin{comment}
%\section{Administraci�n de bases de datos}
\section{Consultas: �lgebra y SQL}

Nombre : \hrulefill\hrulefill \hs\hs DNI: \hrulefill \\

%A partir de la tabla de provincias, autonom�as, poblaci�n y superficie que se ha utilizado en las clases, 
Plantea %�nicamente
el �lgebra 
y el SQL
que resolver�a las siguientes consultas. 
Se puede hacer uso del Access, dbForge o el phpMyAdmin para comprobar los resultados.

\begin{enumerate} \item (4.10.02b) Sacar el listado con los nombres de los clientes, el total pedido y el total pagado por cada uno de ellos.

DGPRODUCTIONS GARDEN	4000,00	6165,00
Gardening Associates	10926,00	10926,00
Gerudo Valley	81849,00	81849,00
Tendo Garden	23794,00	23794,00
Beragua [...] 

\vspace{5cm}

\item (4.10.20) Obtener los nombre de clientes que no est�n al d�a con sus pagos. (Sin excluir los pedidos rechazados)

DGPRODUCTIONS GARDEN

\vspace{5cm}

\item (4.08.09) �Qu� cliente(s) ha rechazado m�s pedidos?

DGPRODUCTIONS GARDEN
Gardening Associates
Gerudo Valley
Tendo Garden
Camunas Jardines S.L.
Flores Marivi
Jardineria Sara
El Jardin Viviente S.L

\vspace{5cm}

\item (4.08.05) Obtener la cantidad facturada con el producto que m�s unidades haya vendido

986,00

\vspace{5cm}

\item (4.08.06) Obtener la cantidad facturada con el producto que m�s dinero haya hecho ingresar

73510,00

\vspace{5cm}

\item (4.10.29) �Cu�l es la descripci�n de la gama del producto que m�s factura? (Sin tener en cuenta devoluciones)

Plantas vistosas para la decoraci�n del jard�n

\vspace{5cm}

\item (4.10.30) �Cu�l es la descripci�n de la gama de productos que m�s factura? (Sin tener en cuenta devoluciones)

Plantas vistosas para la decoraci�n del jard�n

\vspace{5cm}

\item (4.10.33) C�digo de las oficinas que no venden frutales

LON-UK
PAR-FR
TOK-JP

\vspace{5cm}

\item (4.10.35) C�digo del frutal m�s vendido

FR-17

\vspace{5cm}

\item (4.10.37) C�digo del frutal con el que m�s se ha facturado

FR-11

\vspace{5cm}

\item (4.10.38) Frutal con el que m�s se ha ganado (mayores beneficios)

FR-11

\vspace{5cm}

\item (4.08.03b) Obtener los nombres de los clientes que hayan excedido su l�mite de cr�dito

NULL

\vspace{5cm}

\item (4.10.39) Gama de productos m�s rechazada

Frutales

\vspace{5cm}

\item (4.10.43) �Cu�ntos clientes locales y for�neos tiene cada oficina? Indica �nicamente el c�digo de la oficina

BCN-ES	1	9
BOS-USA	0	2
MAD-ES	3	3
PAR-FR	2	0
SFC-USA	1	3
SYD-AU	2	5
TAL-ES	0	5

\vspace{5cm}

\item (4.10.31) Facturaci�n de la gama de productos m�s vendida

140194,00

\vspace{5cm}

\item (4.08.08) �Qu� cliente ha rechazado pedidos por mayor importe?

Gerudo Valley

\vspace{5cm}

\item (4.10.22) �Cu�l ha sido el producto m�s rechazado? Muestra c�digo y nombre

FR-17	Rosal bajo 1ª -En maceta-inicio brotaci�n

\vspace{5cm}

\item (4.10.23) �Cu�l ha sido el producto que m�s dinero nos ha hecho devolver?

OR-247	Trachycarpus Fortunei

\vspace{5cm}

\item (4.10.42) �Qu� ciudad consume m�s frutales? Entendiendo la ciudad del cliente

New York

\vspace{5cm}

\item (4.10.28) �Cu�l es la descripci�n de la gama del producto m�s rechazado?

�rboles peque�os de producci�n frutal

\vspace{5cm}

\item (4.10.25) �Con qu� cliente la empresa gana m�s dinero? (Sin excluir los pedidos rechazados)

Gerudo Valley

\vspace{5cm}

\item (4.10.27) �Cu�l es la descripci�n de la gama de productos m�s rechazada?

�rboles peque�os de producci�n frutal

\vspace{5cm}

\item (4.10.34) Nombre del cliente que m�s frutales ha pedido

Gerudo Valley

\vspace{5cm}

\item (4.10.24) �Con qu� producto la empresa gana m�s dinero? (Sin excluir los pedidos rechazados) Mostrar c�digo y nombre

OR-247	Trachycarpus Fortunei

\vspace{5cm}

\item (4.10.19) Datos de la empresa que m�s ha tardado en pagar su �ltimo pedido (suponiendo que todos est�n al corriente de pagos)

38	El Jardin Viviente S.L	Justin	Smith	2 8005-7161	2 8005-7162	176 Cumberland Street The rocks	(null)	Sydney	Nueva Gales del Sur	Australia	2003	31	800 [...] 

\vspace{5cm}

\item (4.10.26) En qu� ciudad est� la oficina que m�s ha facturado

Boston

\vspace{5cm}

\item (4.10.32) Ciudad de la oficina que m�s frutales vende

Boston

\vspace{5cm}

 \end{enumerate}\newpage

%\end{comment}
%-----------------------------------------------------------


%-----------------------------------------------------------
%\section{Administraci�n de bases de datos}
\section{Soluciones.}
\begin{enumerate} \item (4.10.02b) Sacar el listado con los nombres de los clientes, el total pedido y el total pagado por cada uno de ellos.

DGPRODUCTIONS GARDEN	4000,00	6165,00
Gardening Associates	10926,00	10926,00
Gerudo Valley	81849,00	81849,00
Tendo Garden	23794,00	23794,00
Beragua [...] 



\begin{verbatim} 
SELECT NombreCliente,Pagos,Compras FROM (
    SELECT * FROM (
        SELECT CodigoCliente,SUM(Cantidad) Pagos
          FROM pagos GROUP BY CodigoCliente  
      ) c1 JOIN (
        SELECT CodigoCliente,SUM(Cantidad*PrecioUnidad) Compras
          FROM detallepedidos JOIN pedidos USING(CodigoPedido)
          GROUP BY CodigoCliente  
      ) c2 USING(CodigoCliente)  
  ) c3 JOIN clientes USING(CodigoCliente);
\end{verbatim}

\item (4.10.20) Obtener los nombre de clientes que no est�n al d�a con sus pagos. (Sin excluir los pedidos rechazados)

DGPRODUCTIONS GARDEN



\begin{verbatim} 
SELECT NombreCliente FROM (
    SELECT CodigoCliente FROM (
        SELECT CodigoCliente,SUM(Cantidad*PrecioUnidad) pide
          FROM detallepedidos JOIN pedidos USING(CodigoPedido)
          GROUP BY CodigoCliente  
      ) c1 JOIN (
        SELECT CodigoCliente,SUM(Cantidad) paga FROM pagos GROUP BY CodigoCliente  
      ) c2 USING(CodigoCliente)
      WHERE pide>paga  
  ) c3 JOIN clientes USING(CodigoCliente);
\end{verbatim}

\item (4.08.09) �Qu� cliente(s) ha rechazado m�s pedidos?

DGPRODUCTIONS GARDEN
Gardening Associates
Gerudo Valley
Tendo Garden
Camunas Jardines S.L.
Flores Marivi
Jardineria Sara
El Jardin Viviente S.L



\begin{verbatim} 
SELECT NombreCliente FROM (
    SELECT CodigoCliente
      FROM pedidos WHERE Estado='Rechazado'
      GROUP BY CodigoCliente
      HAVING COUNT(*)=(
        SELECT MAX(n) FROM (
            SELECT CodigoCliente,COUNT(*) n
              FROM pedidos WHERE Estado='Rechazado'
              GROUP BY CodigoCliente  
          ) c1  
      )  
  ) c2 JOIN clientes USING(CodigoCliente);
\end{verbatim}

\item (4.08.05) Obtener la cantidad facturada con el producto que m�s unidades haya vendido

986,00



\begin{verbatim} 
SELECT SUM(Cantidad*PrecioUnidad) total 
  FROM detallepedidos JOIN (
    SELECT CodigoProducto FROM detallepedidos
      GROUP BY CodigoProducto
      HAVING SUM(Cantidad)=(
        SELECT MAX(cantidad) FROM (
            SELECT CodigoProducto,SUM(Cantidad) cantidad
              FROM detallepedidos GROUP BY CodigoProducto  
          ) c1  
      )  
  ) c2 USING(CodigoProducto);
\end{verbatim}

\item (4.08.06) Obtener la cantidad facturada con el producto que m�s dinero haya hecho ingresar

73510,00



\begin{verbatim} 
SELECT SUM(Cantidad*PrecioUnidad) FROM (
    SELECT CodigoProducto FROM detallepedidos 
      GROUP BY CodigoProducto
      HAVING SUM(Cantidad*PrecioUnidad)=(
        SELECT MAX(s) FROM (
            SELECT CodigoProducto,
              SUM(Cantidad*PrecioUnidad) s 
              FROM detallepedidos
              GROUP BY CodigoProducto  
          ) c1
      )  
  ) c2 JOIN detallepedidos USING(CodigoProducto);
\end{verbatim}

\item (4.10.29) �Cu�l es la descripci�n de la gama del producto que m�s factura? (Sin tener en cuenta devoluciones)

Plantas vistosas para la decoraci�n del jard�n



\begin{verbatim} 
SELECT DescripcionTexto FROM (
    SELECT CodigoProducto
      FROM detallepedidos GROUP BY CodigoProducto
      HAVING SUM(Cantidad*PrecioUnidad)=(
        SELECT MAX(s) FROM (
          SELECT CodigoProducto,SUM(Cantidad*PrecioUnidad) s
            FROM detallepedidos GROUP BY CodigoProducto  
          ) c1  
      )  
  ) c2 JOIN productos USING(CodigoProducto)
  JOIN gamasproductos USING(Gama);
\end{verbatim}

\item (4.10.30) �Cu�l es la descripci�n de la gama de productos que m�s factura? (Sin tener en cuenta devoluciones)

Plantas vistosas para la decoraci�n del jard�n



\begin{verbatim} 
SELECT DescripcionTexto FROM (
    SELECT Gama
      FROM detallepedidos JOIN productos USING(CodigoProducto)
      GROUP BY Gama HAVING SUM(Cantidad*PrecioUnidad)=(
        SELECT MAX(s) FROM (
            SELECT Gama,SUM(Cantidad*PrecioUnidad) s
              FROM detallepedidos 
              JOIN productos USING(CodigoProducto)
              GROUP BY Gama  
          ) c1  
      )  
  ) c3 JOIN gamasproductos USING(Gama);
\end{verbatim}

\item (4.10.33) C�digo de las oficinas que no venden frutales

LON-UK
PAR-FR
TOK-JP



\begin{verbatim} 
SELECT o.CodigoOficina FROM oficinas o LEFT JOIN (
    SELECT DISTINCT CodigoOficina FROM (
        SELECT CodigoProducto
          FROM productos WHERE Gama='Frutales'  
      ) c1 JOIN (
        SELECT CodigoProducto,CodigoPedido
          FROM detallepedidos
      ) c2 USING(CodigoProducto) JOIN (
        SELECT CodigoPedido,CodigoCliente FROM pedidos
      ) c3 USING(CodigoPedido) JOIN (
        SELECT CodigoCliente,
          CodigoEmpleadoRepVentas CodigoEmpleado
          FROM clientes
      ) c4 USING(CodigoCliente) JOIN (
        SELECT CodigoEmpleado,CodigoOficina FROM empleados
      ) c5 USING(CodigoEmpleado)  
  ) c2 USING(CodigoOficina)
  WHERE c2.CodigoOficina IS NULL;
\end{verbatim}

\item (4.10.35) C�digo del frutal m�s vendido

FR-17



\begin{verbatim} 
SELECT CodigoProducto FROM (
    SELECT CodigoProducto
      FROM productos WHERE Gama='Frutales'  
  ) c1 JOIN detallepedidos USING(CodigoProducto)
  GROUP BY CodigoProducto
  HAVING SUM(Cantidad)=(
    SELECT MAX(s) FROM (
        SELECT CodigoProducto,SUM(Cantidad) s FROM (
            SELECT CodigoProducto
              FROM productos WHERE Gama='Frutales'  
          ) c1 JOIN detallepedidos USING(CodigoProducto)
          GROUP BY CodigoProducto  
      ) c2  
  );
\end{verbatim}

\item (4.10.37) C�digo del frutal con el que m�s se ha facturado

FR-11



\begin{verbatim} 
SELECT CodigoProducto FROM (
    SELECT CodigoProducto
      FROM productos WHERE Gama='Frutales'  
  ) c1 JOIN detallepedidos USING(CodigoProducto)
  GROUP BY CodigoProducto
  HAVING SUM(Cantidad*PrecioUnidad)=(
    SELECT MAX(total) FROM (
        SELECT CodigoProducto,
          SUM(Cantidad*PrecioUnidad) total FROM (
            SELECT CodigoProducto
              FROM productos WHERE Gama='Frutales'  
          ) c1 JOIN detallepedidos USING(CodigoProducto)
          GROUP BY CodigoProducto
      ) c2  
  );
\end{verbatim}

\item (4.10.38) Frutal con el que m�s se ha ganado (mayores beneficios)

FR-11



\begin{verbatim} 
SELECT CodigoProducto
  FROM (
    SELECT CodigoProducto,
      PrecioVenta-PrecioProveedor margen
      FROM productos WHERE Gama='Frutales'  
  ) c1 JOIN detallepedidos USING(CodigoProducto)
  GROUP BY CodigoProducto
  HAVING SUM(Cantidad*margen)=(
    SELECT MAX(beneficio) FROM (
        SELECT CodigoProducto,
          SUM(Cantidad*margen) beneficio
          FROM (
            SELECT CodigoProducto,
              PrecioVenta-PrecioProveedor margen
              FROM productos WHERE Gama='Frutales'  
          ) c1 JOIN detallepedidos USING(CodigoProducto)
          GROUP BY CodigoProducto  
      ) c2  
  );
\end{verbatim}

\item (4.08.03b) Obtener los nombres de los clientes que hayan excedido su l�mite de cr�dito

NULL



\begin{verbatim} 
SELECT * FROM (
    SELECT CodigoCliente,NombreCliente,total,pago,LimiteCredito limite FROM (
        SELECT CodigoCliente,SUM(total) total FROM (
            SELECT CodigoPedido,SUM(Cantidad*PrecioUnidad) total
              FROM detallepedidos GROUP BY CodigoPedido 
          ) c1 JOIN pedidos USING(CodigoPedido)
          GROUP BY CodigoCliente  
      ) c2 JOIN clientes USING(CodigoCliente)
      JOIN (
        SELECT CodigoCliente,SUM(Cantidad) pago
          FROM pagos GROUP BY CodigoCliente 
      ) c3 USING(CodigoCliente)
  ) c3 WHERE total-pago>limite;
\end{verbatim}

\item (4.10.39) Gama de productos m�s rechazada

Frutales

$c1\gets\prod_{CodigoPedido}(
\sigma_{Estado=Rechazado}(pedidos)
)$

$c3\gets c1\bigotimes_{CodigoPedido}detallepedido
\bigotimes_{CodigoProducto}productos$

$c4\gets_{Gama}\G_{Gama,s\gets suma(Cantidad)}(c3)$

$m\gets\G_{max(s)}(c4)$

$\prod_{Gama}(
\sigma_{s=m}(c4)
)$

\begin{verbatim} 
SELECT Gama FROM (
    SELECT CodigoPedido FROM pedidos WHERE Estado='Rechazado'  
  ) c1 JOIN detallepedidos USING(CodigoPedido)
  JOIN productos USING(CodigoProducto)
  GROUP BY Gama
  HAVING SUM(Cantidad)=(
    SELECT MAX(s) FROM (
        SELECT Gama,SUM(Cantidad) s FROM (
            SELECT CodigoPedido FROM pedidos
              WHERE Estado='Rechazado'  
          ) c1 JOIN detallepedidos USING(CodigoPedido)
          JOIN productos USING(CodigoProducto)
          GROUP BY Gama  
      ) c4  
  );
\end{verbatim}

\item (4.10.43) �Cu�ntos clientes locales y for�neos tiene cada oficina? Indica �nicamente el c�digo de la oficina

BCN-ES	1	9
BOS-USA	0	2
MAD-ES	3	3
PAR-FR	2	0
SFC-USA	1	3
SYD-AU	2	5
TAL-ES	0	5



\begin{verbatim} 
SELECT CodigoOficina, 
  IFNULL(n_local,0) n_local,
  n_total-IFNULL(n_local,0) n_foraneos
  FROM (
    SELECT CodigoOficina,COUNT(*) n_total FROM oficinas
      JOIN empleados USING(CodigoOficina)
      JOIN clientes
      ON empleados.CodigoEmpleado=clientes.CodigoEmpleadoRepVentas
      GROUP BY CodigoOficina  
  ) c2 LEFT JOIN (
    SELECT CodigoOficina,COUNT(*) n_local FROM oficinas
      JOIN empleados USING(CodigoOficina)
      JOIN clientes
      ON empleados.CodigoEmpleado=clientes.CodigoEmpleadoRepVentas
      WHERE oficinas.Ciudad=clientes.Ciudad
      GROUP BY CodigoOficina  
  ) c1 USING(CodigoOficina);
\end{verbatim}

\item (4.10.31) Facturaci�n de la gama de productos m�s vendida

140194,00

$c1\gets_{Gama}\G_{Gama,s\gets suma(Cantidad)}(
detallepedidos\bigotimes_{CodidoProducto}productos)$

$m\gets\G_{max(s)}(c1)$

$\prod_{Gama}(\sigma_{s=m}(c1))$

$c3\gets\prod_{CodigoProducto}(
c2\bigotimes_{Gama}productos
)$

$\G_{suma(Cantidad*PrecioUnidad)}(
c3\bigotimes_{CodigoPedido}detallepedido
)$

\begin{verbatim} 
SELECT SUM(Cantidad*PrecioUnidad) FROM (
    SELECT CodigoProducto FROM (
        SELECT Gama
          FROM detallepedidos JOIN  productos
          USING(CodigoProducto) GROUP BY Gama
          HAVING SUM(Cantidad)=(
            SELECT MAX(s) FROM (
                SELECT Gama,SUM(Cantidad) s
                  FROM detallepedidos JOIN  productos
                  USING(CodigoProducto) GROUP BY Gama  
              ) c1
          )  
      ) c2 JOIN productos USING(Gama)  
  ) c3 JOIN detallepedidos USING(CodigoProducto);
\end{verbatim}

\item (4.08.08) �Qu� cliente ha rechazado pedidos por mayor importe?

Gerudo Valley



\begin{verbatim} 
SELECT NombreCliente FROM (
    SELECT CodigoCliente FROM (
        SELECT CodigoPedido,CodigoCliente FROM pedidos
          WHERE Estado='Rechazado'  
      ) c1 JOIN detallepedidos USING(CodigoPedido)
      GROUP BY CodigoCliente
      HAVING SUM(Cantidad*PrecioUnidad)=(
        SELECT MAX(s) FROM (
            SELECT CodigoCliente,SUM(Cantidad*PrecioUnidad) s
              FROM (
                SELECT CodigoPedido,CodigoCliente
                  FROM pedidos WHERE Estado='Rechazado'  
              ) c1 JOIN detallepedidos USING(CodigoPedido)
              GROUP BY CodigoCliente  
          ) c2  
      )  
  ) c3 JOIN clientes USING(CodigoCliente);
\end{verbatim}

\item (4.10.22) �Cu�l ha sido el producto m�s rechazado? Muestra c�digo y nombre

FR-17	Rosal bajo 1ª -En maceta-inicio brotaci�n



\begin{verbatim} 
SELECT CodigoProducto,Nombre FROM (
    SELECT CodigoProducto FROM (
        SELECT CodigoPedido FROM pedidos WHERE Estado='Rechazado'  
      ) c1 JOIN detallepedidos USING(CodigoPedido)
      GROUP BY CodigoProducto
      HAVING SUM(Cantidad)=(
        SELECT MAX(s) FROM (
            SELECT CodigoProducto,SUM(Cantidad) s FROM (
                SELECT CodigoPedido FROM pedidos
                  WHERE Estado='Rechazado'  
              ) c1 JOIN detallepedidos USING(CodigoPedido)
              GROUP BY CodigoProducto  
          ) c2  
      )  
  ) c3 JOIN productos USING(CodigoProducto);
\end{verbatim}

\item (4.10.23) �Cu�l ha sido el producto que m�s dinero nos ha hecho devolver?

OR-247	Trachycarpus Fortunei



\begin{verbatim} 
SELECT CodigoProducto,Nombre FROM (
    SELECT CodigoProducto FROM (
        SELECT CodigoPedido FROM pedidos
          WHERE Estado='Rechazado'  
      ) c1 JOIN detallepedidos USING(CodigoPedido)
      GROUP BY CodigoProducto
      HAVING SUM(Cantidad*PrecioUnidad)=(
        SELECT MAX(s) FROM (
            SELECT CodigoProducto,
              SUM(Cantidad*PrecioUnidad) s FROM (
                SELECT CodigoPedido FROM pedidos
                  WHERE Estado='Rechazado'  
              ) c1 JOIN detallepedidos USING(CodigoPedido)
              GROUP BY CodigoProducto  
          ) c2  
      )  
  ) c3 JOIN productos USING(CodigoProducto);
\end{verbatim}

\item (4.10.42) �Qu� ciudad consume m�s frutales? Entendiendo la ciudad del cliente

New York



\begin{verbatim} 
SELECT Ciudad FROM (
    SELECT CodigoProducto
      FROM productos WHERE Gama='Frutales'  
  ) c1 JOIN detallepedidos USING(CodigoProducto)
  JOIN pedidos USING(CodigoPedido)
  JOIN clientes USING(CodigoCliente)
  GROUP BY Ciudad
  HAVING SUM(Cantidad)=(
    SELECT MAX(s) FROM (
        SELECT Ciudad,SUM(Cantidad) s FROM (
            SELECT CodigoProducto
              FROM productos WHERE Gama='Frutales'  
          ) c1 JOIN detallepedidos USING(CodigoProducto)
          JOIN pedidos USING(CodigoPedido)
          JOIN clientes USING(CodigoCliente)
          GROUP BY Ciudad  
      ) c2  
  );
\end{verbatim}

\item (4.10.28) �Cu�l es la descripci�n de la gama del producto m�s rechazado?

�rboles peque�os de producci�n frutal

$c1\gets\prod_{CodigoPedido}(
\sigma_{Estado=Rechazado}(pedidos)
)$

$c2\gets_{CodigoProducto}\G_{CodigoProducto,s\gets suma(Cantidad)}(
c1\bigotimes_{CodigoPedido}detallepedido
)$

$m\gets\G_{max(s)}(c2)$

$c3\gets\prod_{CodigoProducto}(
\sigma_{s=m}(c2)
)$

$c4\gets\prod_{Gama}(
c3\bigotimes_{CodigoProducto}productos
)$

$\prod_{DescripcionTexto}(
c4\bigotimes_{Gama}gamaproductos
)$

\begin{verbatim} 
SELECT DescripcionTexto FROM (
    SELECT CodigoProducto FROM (
        SELECT CodigoPedido FROM pedidos
          WHERE Estado='Rechazado'  
      ) c1 JOIN detallepedidos USING(CodigoPedido)
      GROUP BY CodigoProducto
      HAVING SUM(Cantidad)=(
        SELECT MAX(s) FROM (
            SELECT CodigoProducto,SUM(Cantidad) s FROM (
                SELECT CodigoPedido FROM pedidos
                  WHERE Estado='Rechazado'  
              ) c1 JOIN detallepedidos USING(CodigoPedido)
              GROUP BY CodigoProducto  
          ) c2  
      )  
  ) c4 JOIN productos USING(CodigoProducto)
  JOIN gamasproductos USING(Gama);
\end{verbatim}

\item (4.10.25) �Con qu� cliente la empresa gana m�s dinero? (Sin excluir los pedidos rechazados)

Gerudo Valley



\begin{verbatim} 
SELECT NombreCliente FROM (
    SELECT CodigoCliente FROM (
        SELECT CodigoProducto,
          PrecioVenta-PrecioProveedor margen
          FROM productos  
      ) c1 JOIN detallepedidos USING(CodigoProducto)
      JOIN pedidos USING(CodigoPedido)
      GROUP BY CodigoCliente
      HAVING SUM(Cantidad*margen)=(
        SELECT MAX(s) m FROM (
            SELECT CodigoCliente,
              SUM(Cantidad*margen) s FROM (
                SELECT CodigoProducto,
                  PrecioVenta-PrecioProveedor margen
                  FROM productos  
              ) c1 JOIN detallepedidos USING(CodigoProducto)
              JOIN pedidos USING(CodigoPedido)
              GROUP BY CodigoCliente  
          ) c2  
      )  
  ) c3 JOIN clientes USING(CodigoCliente);
\end{verbatim}

\item (4.10.27) �Cu�l es la descripci�n de la gama de productos m�s rechazada?

�rboles peque�os de producci�n frutal



\begin{verbatim} 
SELECT DescripcionTexto FROM (
    SELECT Gama FROM (
        SELECT CodigoPedido
          FROM pedidos WHERE Estado='Rechazado'  
      ) c1 JOIN detallepedidos USING(CodigoPedido)
      JOIN productos USING(CodigoProducto)
      GROUP BY Gama
      HAVING SUM(Cantidad)=(
        SELECT MAX(s) FROM (
            SELECT Gama,SUM(Cantidad) s FROM (
                SELECT CodigoPedido
                  FROM pedidos WHERE Estado='Rechazado'  
              ) c1 JOIN detallepedidos USING(CodigoPedido)
              JOIN productos USING(CodigoProducto)
              GROUP BY Gama  
          ) c2  
      )  
  ) c4 JOIN gamasproductos USING(Gama);
\end{verbatim}

\item (4.10.34) Nombre del cliente que m�s frutales ha pedido

Gerudo Valley



\begin{verbatim} 
SELECT NombreCliente FROM (
    SELECT CodigoCliente FROM (
        SELECT CodigoProducto FROM productos
          WHERE Gama='Frutales'  
      ) c1 JOIN detallepedidos USING(CodigoProducto)
      JOIN pedidos USING(CodigoPedido)
      GROUP BY CodigoCliente
      HAVING SUM(Cantidad)=(
        SELECT MAX(s) FROM (
            SELECT CodigoCliente,SUM(Cantidad) s FROM (
                SELECT CodigoProducto FROM productos
                  WHERE Gama='Frutales'  
              ) c1 JOIN detallepedidos 
              USING(CodigoProducto)
              JOIN pedidos USING(CodigoPedido)
              GROUP BY CodigoCliente  
          ) c2  
      )  
  ) c3 JOIN clientes USING(CodigoCliente);
\end{verbatim}

\item (4.10.24) �Con qu� producto la empresa gana m�s dinero? (Sin excluir los pedidos rechazados) Mostrar c�digo y nombre

OR-247	Trachycarpus Fortunei



\begin{verbatim} 
SELECT CodigoProducto,Nombre FROM (
    SELECT CodigoProducto FROM (
        SELECT CodigoProducto,SUM(Cantidad) s
          FROM detallepedidos
          GROUP BY CodigoProducto  
      ) c1 JOIN (
        SELECT CodigoProducto,
          PrecioVenta-PrecioProveedor margen
          FROM productos  
      ) c2 USING(CodigoProducto)
      WHERE s*margen=(
        SELECT MAX(beneficio) FROM (
            SELECT CodigoProducto,s*margen beneficio FROM (
                SELECT CodigoProducto,SUM(Cantidad) s
                  FROM detallepedidos
                  GROUP BY CodigoProducto  
              ) c1 JOIN (
                SELECT CodigoProducto,
                  PrecioVenta-PrecioProveedor margen
                  FROM productos  
              ) c2 USING(CodigoProducto)  
          ) c3  
      )  
  ) c4 JOIN productos USING(CodigoProducto);
\end{verbatim}

\item (4.10.19) Datos de la empresa que m�s ha tardado en pagar su �ltimo pedido (suponiendo que todos est�n al corriente de pagos)

38	El Jardin Viviente S.L	Justin	Smith	2 8005-7161	2 8005-7162	176 Cumberland Street The rocks	(null)	Sydney	Nueva Gales del Sur	Australia	2003	31	800 [...] 



\begin{verbatim} 
SELECT * FROM (
    SELECT CodigoCliente FROM (
        SELECT CodigoCliente,MAX(FechaPago) paga 
          FROM pagos GROUP BY CodigoCliente  
      ) c1 JOIN (
        SELECT CodigoCliente,MAX(FechaEntrega) compra 
          FROM pedidos GROUP BY CodigoCliente
      ) c2 USING(CodigoCliente)  
      WHERE DATEDIFF(paga,compra)=(
        SELECT MIN(retardo) FROM (
            SELECT CodigoCliente,DATEDIFF(paga,compra) retardo 
              FROM (
                SELECT CodigoCliente,MAX(FechaPago) paga 
                  FROM pagos GROUP BY CodigoCliente  
              ) c1 JOIN (
                SELECT CodigoCliente,MAX(FechaEntrega) compra 
                  FROM pedidos GROUP BY CodigoCliente
              ) c2 USING(CodigoCliente)  
          ) c3  
      )  
  ) c5 JOIN clientes USING(CodigoCliente);
\end{verbatim}

\item (4.10.26) En qu� ciudad est� la oficina que m�s ha facturado

Boston



\begin{verbatim} 
SELECT Ciudad
  FROM (
    SELECT CodigoOficina,Ciudad FROM oficinas
  ) c1 JOIN (
    SELECT CodigoOficina,CodigoEmpleado FROM empleados
  ) c2 USING(CodigoOficina) JOIN (
    SELECT CodigoCliente,CodigoEmpleadoRepVentas
      FROM clientes  
  ) c3 ON CodigoEmpleadoRepVentas=CodigoEmpleado JOIN (
    SELECT CodigoPedido,CodigoCliente FROM pedidos  
  ) c4 USING(CodigoCliente) JOIN (
    SELECT CodigoPedido,Cantidad,PrecioUnidad
      FROM detallepedidos
  ) c5 USING(CodigoPedido)
  GROUP BY Ciudad,CodigoOficina
  HAVING SUM(Cantidad*PrecioUnidad)=(
    SELECT MAX(s) FROM (
        SELECT CodigoOficina,SUM(Cantidad*PrecioUnidad) s
          FROM (
            SELECT CodigoOficina,Ciudad FROM oficinas
          ) c1 JOIN (
            SELECT CodigoOficina,CodigoEmpleado FROM empleados
          ) c2 USING(CodigoOficina) JOIN (
            SELECT CodigoCliente,CodigoEmpleadoRepVentas
              FROM clientes  
          ) c3 ON CodigoEmpleadoRepVentas=CodigoEmpleado JOIN (
            SELECT CodigoPedido,CodigoCliente FROM pedidos  
          ) c4 USING(CodigoCliente) JOIN (
            SELECT CodigoPedido,Cantidad,PrecioUnidad
              FROM detallepedidos
          ) c5 USING(CodigoPedido)
          GROUP BY CodigoOficina  
      ) c6  
  );
\end{verbatim}

\item (4.10.32) Ciudad de la oficina que m�s frutales vende

Boston

$c1\gets\prod_{CodigoProducto}(
\sigma_{Gama=Frutales}(productos)
)$

$c2\gets\prod_{CodigoProducto,CodigoPedido}(detalleproducto)$

$c3\gets\prod_{CodigoPedido,CodigoCliente}(pedidos)$

$c4\gets\prod_{CodigoCliente,
CodigoEmpleado\gets CodigoEmpleadoRepVentas
}(clientes)$

$c5\gets\prod_{CodigoEmpleado,CodigoOficina}(empleados)$

$c6\gets\prod_{CodigoOficina,Ciudad}(oficinas)$

$c7\gets 
c1\bigotimes_{CodigoProducto}c2
\bigotimes_{CodigoPedido}c3
\bigotimes_{CodigoCliente}c4
\bigotimes_{CodigoEmpleado}c5
\bigotimes_{CodigoOficina}c6
$

$c8\gets
_{Ciudad,CodigoOficina}\G_{Ciudad,CodigoOficina,suma(Cantidad)}(c7)$

$m\gets\G_{max(s)}(c8)$

$\prod_{Ciudad}(
\sigma_{s=m}(c8)
)$

\begin{verbatim} 
SELECT Ciudad FROM (
    SELECT CodigoProducto
      FROM productos WHERE Gama='Frutales'  
  ) c1 JOIN (
    SELECT CodigoProducto,CodigoPedido,Cantidad
      FROM detallepedidos
  ) c2 USING(CodigoProducto) JOIN (
    SELECT CodigoPedido,CodigoCliente FROM pedidos  
  ) c3 USING(CodigoPedido) JOIN (
    SELECT CodigoCliente,
      CodigoEmpleadoRepVentas CodigoEmpleado
      FROM clientes
  ) c4 USING(CodigoCliente) JOIN (
    SELECT CodigoEmpleado,CodigoOficina FROM empleados
  ) c5 USING(CodigoEmpleado) JOIN (
    SELECT CodigoOficina,Ciudad FROM oficinas
  ) c6 USING(CodigoOficina)
  GROUP BY Ciudad,CodigoOficina
  HAVING SUM(Cantidad)=(
    SELECT MAX(s) FROM (
        SELECT CodigoOficina,SUM(Cantidad) s FROM (
            SELECT CodigoProducto
              FROM productos WHERE Gama='Frutales'  
          ) c1 JOIN (
            SELECT CodigoProducto,CodigoPedido,Cantidad
              FROM detallepedidos
          ) c2 USING(CodigoProducto) JOIN (
            SELECT CodigoPedido,CodigoCliente FROM pedidos  
          ) c3 USING(CodigoPedido) JOIN (
            SELECT CodigoCliente,
              CodigoEmpleadoRepVentas CodigoEmpleado
              FROM clientes
          ) c4 USING(CodigoCliente) JOIN (
            SELECT CodigoEmpleado,CodigoOficina FROM empleados
          ) c5 USING(CodigoEmpleado) JOIN (
            SELECT CodigoOficina,Ciudad FROM oficinas
          ) c6 USING(CodigoOficina)
          GROUP BY CodigoOficina  
      ) c2  
  );
\end{verbatim}

 \end{enumerate}\newpage
%-----------------------------------------------------------
	
	
%\includepdf[pages=-,offset=0mm 0]{fondo}

%\ClearShipoutPicture
%\includepdf[pages=-,offset=14mm 0]{Ord_201_18783_25032015_fdo}
	
	
%$ \vee \left(  \sqsupset\times\sqsubset \uprho \leftarrow \uptheta \bowtie \land \textrm{casa \c{G}} \prod \updelta \times \cup \cap \bigcup \bigcap \bigvee \bigwedge \bigodot \bigotimes \bigoplus   \right) \Join \ljoin \rjoin \fjoin $


\end{document}