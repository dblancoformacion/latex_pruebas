\documentclass[12pt]{article}
\usepackage[latin1]{inputenc}
\usepackage[spanish]{babel}
\usepackage{a4wide}

\begin{document}

\title{Git, inicio r�pido}
\author{David, Alpe Formaci�n}

\maketitle
\tableofcontents

\section{�Por d�nde empiezo?}

\subsection{clone}
\label{sec:clone}

La forma m�s sencilla de empezar a trabajar con $git$ es clonando un repositorio existente con el comando $clone$\footnote{
Si a�n no has instalado el $git$ en tu equipo, visita https://git-scm.com/, inst�late el software y, con el bot�n derecho en el escritorio o en cualquier otra carpeta en la que quieras trabajar con el repositorio a descargar, selecciona la opci�n Git Bash here y se abrir� el int�rprete de comandos en el que podr�s introducir los comandos que comentamos en este documento.
}

\begin{verbatim}
$ git clone https://github.com/dblancoformacion/latex_pruebas
\end{verbatim}

Ver�s que se crea una carpeta nueva en el directorio en el que est�s trabajando. Si accedes al directorio latex\_pruebas con

\begin{verbatim}
$ cd latex_pruebas
\end{verbatim}

ya tendr�s control de tu copia local del repositorio.

\subsection{log}

El historial de cambios que se han realizado en el proyecto se pueden visualizar con:
%$commits$\footnote{Un $commit$ es una actualizaci�n de los cambios realizados en el repositorio local}


\begin{verbatim}
$ git log --oneline --all
\end{verbatim}

Obtendr�s algo parecido a esto:

\begin{verbatim}
89a7615 (HEAD -> master, origin/master, origin/HEAD) Git push
7f84d5a Segunda modificaci�n
881a3bd Inicio
\end{verbatim}

\subsection{pull}

Para actualizar nuestro repositorio local, utilizamos el comando $pull$

\begin{verbatim}
$ git pull
\end{verbatim}

\section{Viajado en el tiempo}

Hasta el momento, continuamos trabajando en el repositorio de otra persona, a�n no estamos registrando nuestros propios cambios.

Que no los registremos, no impide que podamos realizarlos. Un repositorio descargado en nuestro equipo es perfectamente posible modificarlo.

\subsection{diff}

Con el comando \textit{diff} podremos visualizar los cambios realizados en los ficheros del repositorio\footnote{Si a�adimos ficheros nuevos en el directorio de trabajo, \textit{diff} no los visualizar� hasta que se lo indiquemos con un $add$.} desde el �ltimo $pull$

\begin{verbatim}
$ git diff
\end{verbatim}

\subsection{reset}

Para descartar todos los cambios realizados y volver a la versi�n almacenada en el repositorio, tenemos el comando $reset$

\begin{verbatim}
$ git reset --hard
\end{verbatim}

\subsection{checkout}




\section{Creando nuestro propio repositorio}

Hasta ahora, �nicamente hemos visualizado repositorios de otros autores. 
Aunque es posible trabajar en un repositorio local para llevar un control de versiones de nuestros propios desarrollos, haremos uso de un repositorio en la nube para poder compartir nuestro trabajo.

Para poder subir nuestros contenidos a internet, debemos crear una cuenta de usuario en un alojamiento de repositorios en la nube como github.com o gitlab.com

Una vez creada la cuenta y generado un nuevo repositorio, podremos clonarlo en nuestro equipo y trabajar con una copia de repositorio en local con el comando $clone$ como se ha visto en el apartado \ref{sec:clone}, p�g. \pageref{sec:clone}.

\subsection{add}

De momento, el repositorio estar� vac�o. S�lo contendr� la carpeta oculta .git, pero ya tendr� configurada la direcci�n del repositorio remoto y otros par�metros que, siguiendo estas instrucciones, no tendremos que configurar manualmente.

Podemos crear o copiar cualquier contenido en la carpeta de trabajo.

Para que $git$ pueda visualizar los cambios realizados en un determinado fichero, debemos pedirle que lo tenga en cuenta\footnote{$Stage$} con el comando $add$ y su nombre o, con un punto, para que realice el seguimiento de todos lo ficheros del directorio

\begin{verbatim}
$ git add .
\end{verbatim}

\subsection{commit}

\subsection{push}

\section{Cuando las cosas se complican}

\subsection{branch}

\subsection{merge}
 
\end{document}